% !TeX spellcheck = pl_PL
\documentclass[10pt,a4paper]{article}

\usepackage{geometry}
\geometry{
	a4paper,
	total={170mm,257mm},
	left=20mm,
	top=20mm,
}


% większość można wywalić, przekopiowane z projektu z MNUMów
\usepackage{mathtools}
\usepackage{polski}
\usepackage{amsfonts}
\usepackage{amsmath}
\usepackage[T1]{fontenc}
\usepackage{listings}
\usepackage{booktabs}
\usepackage{csvsimple}
\usepackage{geometry}
\usepackage{siunitx}
\usepackage{caption}
\usepackage{booktabs}
\usepackage{longtable}
\usepackage{float}
\usepackage{algpseudocode}
\usepackage{algorithm}
\usepackage{csquotes}
\usepackage{hyperref}



\captionsetup[table]{skip=10pt}

\sisetup{round-mode=places, round-precision=3, round-integer-to-decimal, scientific-notation = true, table-format = 1.3e2, table-number-alignment=center, group-separator={,}}


\title{Inżynieria uczenia maszynowego - projekt}
\author{Tomasz Owienko \and Anna Schäfer}
\date{29.11.2023}


% definicja problemu biznesowego
% zadania modelowania
% założenia
% kryteria sukcesu
% analiza danych z perspektywy realizacji zadań


\begin{document}
\maketitle

\section{Temat projektu}




Temat projektu przekazany przez Klienta:

\begin{displayquote}
	\textit{Może bylibyśmy w stanie wygenerować playlistę, która spodoba się kilku wybranym osobom jednocześnie? Coraz więcej osób używa Pozytywki podczas różnego rodzaju imprez i taka funkcjonalność byłaby hitem!}
\end{displayquote}


\section{Problem biznesowy}

\textbf{TODO} coś o samej Pozytywce	w ramach kontekstu

Celem projektu jest realizacja funkcjonalności pozwalającej użytkownikom serwisu Pozytywka na generowanie playlist, których z których utwory podobać się będą wybranej grupie użytkowników. Taka funkcjonalność mogłaby być wykorzystywana do automatycznego układania playlist na imprezy w taki sposób, aby ich zawartość trafiała w gust jak największej części odbiorców. Implementacja takiej funkcjonalności ma zwiększyć zadowolenie użytkowników z jakości playlist odtwarzanych na imprezach, tym samym zwiększając ich zadowolenie z użytkowania portalu.

\subsection*{Biznesowe kryterium sukcesu}
\textbf{TODO możliwości - wybrać jakiś konkret:}
\begin{itemize}
\item Dla $k\%$ użytkowników, playlista odtwarzana przez co najmniej $X\%$ jej czasu trwania w ciągu jednej z najbliższych $N$ sesji użytkownika 
\item Dla $k\%$ użytkowników, podczas odtwarzania playlisty przez co najmniej $X\%$ jej czasu trwania pomijane jest co najwyżej $Y\%$ utworów
\item Coś jeszcze?
\end{itemize}



\subsection{Założenia}
\begin{itemize}
	\item Playlisty generowane będą na podstawie profili oraz historii sesji nie więcej niż $N$ użytkowników 
	
	\textbf{TODO ile to N}
	
	\item Playlisty w większości przypadków użycia nie będą wykorzystywane wielokrotnie
	\item Dobór kolejności utworów na playliście nie jest przedmiotem zadania
	\item Dostęp do playlisty mają wszyscy użytkownicy, których profile i historia sesji były uwzględnione przy jej generowaniu / tylko użytkownik, który utworzył playlistę
\end{itemize}

\subsection{Pożądane cechy rozwiązania}
\begin{itemize}

\item Playlista może być wygenerowana w bardzo krótkim czasie

\item Funkcjonalność zachowuje się poprawnie dla nowo dodanych użytkowników oraz utworów

\textbf{TODO cold start - czy istotny?}

\item W ocenianiu gustu muzycznego poszczególnych użytkowników większe znaczenie powinny mieć niedawno odtwarzane utwory

\end{itemize}

\section{Zadanie modelowania}
	
Projekt zakłada zamodelowanie problemu jako zadanie generowania rekomendacji. Planowane jest zastosowanie podejścia \textit{collaborative filtering}, które (w kontekście zadania) opiera się na wyszukiwaniu użytkowników podobnych do rozpatrywanych i generowania rekomendacji w oparciu o ich historie sesji. Do realizacji podejścia \textit{collaborative filtering} zastosowana zostanie technika rozkładu macierzy interakcji między użytkownikami, a utworami. Przewidziane jest porównanie jakości modelu korzystającego z macierzy \textit{feedbacku niejawnego} (użytkownik $X$ odtworzył utwór $Y$), oraz \textit{feedbacku jawnego} (użytkownik $X$ wystawił utworowi $Y$ ocenę $Z$).

Podejście \textit{collaborative filtering} pozwala na generowanie rekomendacji dla pojedynczego użytkownika. Aby dostosować je do problemu, generowanie rekomendacji dla wielu użytkowników jednocześnie zamodelowane będzie jako:
\begin{enumerate}

\item Wygenerowanie bardzo dużej liczby rekomendacji dla każdego z użytkowników wraz z oceną podobieństwa poszczególnych utworów do gustu muzycznego użytkownika (liczba rekomendacji proporcjonalna do liczby użytkowników)
\item Znormalizowanie ocen podobieństwa do zakresu $<0, 1>$
\item Wyznacznie zbioru utworów, które pojawiły się w rekomendacjach wszystkich użytkowników
\item Wybranie tych utworów, dla których iloczyn ich ocen dla wszystkich użytkowników jest największy

\end{enumerate}

Istotnym problemem w rozważanym zadaniu jest tzw. \textit{cold-start}, czyli zachowanie modelu dla użytkowników bądź utworów, na których model nie był trenowany. Tradycyjne podejścia rozkładu macierzy interakcji takie jak \textit{Funk MF} czy \textit{SVD++} nie przewidują występowania takich sytuacji i spisują się słabo w scenariuszach \textit{cold-start}. W rozwiązaniu zostanie wykorzystany model \href{https://arxiv.org/pdf/1507.08439.pdf}{LightFM}, który rozwiązuje problem \textit{cold-start} przez zastosowanie metadanych (atrybutów) do opisywania zarówno użytkowników, jak i utworów. Przykładowo, nowy użytkownik dodany do systemu nie posiada historii sesji, ale jest znany jego wiek, płeć, czy chociażby zadeklarowane preferencje, co do gatunków muzycznych - dane te można wykorzystać do wyszukiwania podobieństw względem innych użytkowników.

\section{Analiza dostarczonych danych}

\textbf{TODO kolejne iteracja zbierania danych od Klienta, dodatkowe wymagania od zadania (liczba użytkowników, więcej historii sesji)}

\textbf{TODO - czy mamy opisywać cały feature engineering?}

Z perspektywy wytestowania podejścia opartego na \textit{jawnym feedbacku} należy rozważyć sposób zamodelowania takiej informacji - nie jest ona dana bezpośrednio w zbiorze danych. Jednym z możliwych podejść może być wyznaczenie domniemanej oceny wystawionej przez użytkownika danemu utworowi na podstawie:

\begin{itemize}

\item Częstotliwości odtwarzania danego utworu
\item Częstotliwości występowania zdarzenia \texttt{like}
\item Częstotliwości występowania zdarzenia \texttt{skip}

\end{itemize}
	
	

\end{document}
